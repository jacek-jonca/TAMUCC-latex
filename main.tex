\documentclass[pagesize,DIV14]{scrartcl}

%% Fontspec provides an interface to system fonts and Opentype features
\usepackage{fontspec}

 %% Load default features. An asterisk means they are supposed to be
 %% enabled by default, but engines may behave differently.
 \defaultfontfeatures{%
  RawFeature={%
    %+calt   % *Contextual alternates
    %,+clig  % *contextual ligatures
    %,+ccmp  % *composition & decomposition
    %,+tlig  % 'tex-ligatures': `` '' -- --- !` ?` << >>
    ,+cv06  % narrow guillemets
    }%
  }

 %% Finally set the main font
 \setmainfont{EB Garamond}

%% Load Microtype with default settings. This will use the
%% EB-Garamond protrusion definitions if present.
\usepackage{microtype}

%% Use polyglossia to handle different languages.
\usepackage[%
  babelshorthands % needed to access "| as a ZWNJ symbol
  ]{polyglossia}
 \setdefaultlanguage{english}
 \setotherlanguages{%
  french,czech,polish,latin,italian,catalan,german,romanian,%
  portuges,swedish,finnish,dutch,russian,serbian,ukrainian,%
  bulgarian,slovenian,vietnamese,greek%
  }

%% csquotes provides an intelligent for quotation marks
\usepackage{csquotes}

%% Realscripts allows to easily access the font’s own sub-/superscripts
%% with \textsuperscript and \textsubscript as well as \textsubsuperscript
\usepackage{realscripts}

%% Multicol for the two-column samples
\usepackage{multicol}

%% Format the headings in smallcaps
\setkomafont{disposition}{\normalcolor\normalfont\rmfamily\scshape}

%% Add page numbers
\pagenumbering{arabic}

\begin{document}
{\pagestyle{empty}
 \begin{center}
  \pagestyle{empty}
  \vspace*{5cm}
  \fontsize{48}{48}{\addfontfeature{Color=980000}
   EB Garamond}\\
  \vspace*{2cm}
  {\fontsize{24}{24}\addfontfeature{Color=000000}\textit{Claude {\addfontfeature{RawFeature=+swsh}G}aramont’s designs go open source}
  \vfill\vfill
  \today\\\vspace*{2cm}}%<-- superfluous brace. Don’t know, why it’s needed.
  \fontsize{48}{48}{\addfontfeature{Color=980000}
  \char"E001 \char"E002}
 \end{center}\clearpage
}

\section{Samples}
%% Samples that show the font used in different languages.
\begin{multicols}{2}

%% English
\paragraph*{Texas A\&M University -- Corpus Christi} is one of America’s best-kept secrets. Resting on its own island in the heart of the Texas Rivera we are a window to the tropics. We are an exotic land of sunshine, palm trees, and gulf breeze. Surrounded by the Oso and Corpus Christi Bays, we are minutes from the picturesque beaches of Padre and Mustang Islands, and the bustling downtown of Corpus Christi. If you enjoy abundant sunshine, all-year surfing, kayaking, and fishing, then the Island University is your own paradise.

\textit{Garamond came to prominence in 1541, when three of his Greek typefaces (e.g. the Grecs du roi (1541)) were requested for a royally-ordered book series by Robert Estienne. Garamond based these types on the handwriting of Angelo Vergecio, the King’s Librarian at Fontainebleau, as well as that of his ten-year-old pupil, Henri Estienne. According to Arthur Tilley, the resulting books are “among the most finished specimens of typography that exist.” Shortly thereafter, Garamond created the Roman types for which […]}\\
{\scriptsize From Wikipedia, the free encyclopedia}

%% German
\begin{german}
%% Use the "locl" feature from the font:
\addfontfeature{Language=German}
\paragraph*{Claude Garamond,} auch Garamont, (*1499 (oder 1490) in Paris; †\,1561 in Paris) war ein französischer Schriftgießer, Typograf, Stempelschneider und Verleger. Er schuf die noch heute verwendete Schriftart Garamond.

Claude Garamond lernte das Handwerk des Schriftschneidens bei Geoffroy Tory, er war Schüler und Mitarbeiter des Pariser Stempelschneiders und Druckers Antoine Augereau, einige der Arbeiten Garamonds werden in verschiedenen Quellen Augereau zugeordnet – in jedem Fall ist eine Einflussnahme seines Lehrers bis zu dessen Tod auf dem Scheiterhaufen im Jahr 1534 sehr wahrscheinlich.

\textit{Garamonds erste Antiqua-Schriften dürf"|ten um 1530/1531 entstanden sein, als erstmals in vier verschiedenen Pariser Druckereien Schriften eines neuen Typus auf"|tauchten, die Claude Garamond zugeschrieben werden konnten. Die Vorarbeiten hierzu sind bis ins Jahr 1525 zurückzudatieren.}\\
{\scriptsize Aus Wikipedia, der freien Enzyklopädie}
\end{german}


\end{multicols}

\clearpage

%% Documentation of the Opentype features
\section{Features}

The Opentype specs recommend some features to be enabled by default. As this is not mandatory, engines may decide to leave it to the user to activate them. Features that should be on by default are marked with an asterisk (*). All other features have to be enabled when they are required.

\subsection{Locals}
\paragraph*{German} prevents f-ligatures that are usually not allowed like fb, fh, fj, ffb, etc. because they would only occur at \enquote{Wortfugen}: {\addfontfeature{Language=German}Laufband, aufjagen, Sauerstoffflasche,…}\par
In general it has to be noted that in German typography, ligatures are a complex matter, tied to the language’s morphology. There’s no way to apply them automatically without manual intervention. Efforts have been made to alleviate this task, like Mico Loretan’s \textsc{selnolig}\footnote{https://github.com/micoloretan/selnolig} for LuaLaTeX. Still, if in doubt, better leave the ligatures away!
\paragraph*{Turkish, Azerbaijani, Crimtatar} correct replacement of i/ı as smallcaps etc.
\paragraph*{Latin} replaces U by V, initial u by v and non-initial v by u: universalis → {\addfontfeature{Language=Latin} universalis}
\paragraph*{Catalan} Correct positioning of “·” between two “l”s: {\addfontfeature{Language=Catalan}PARAL·LEL, paral·lel, \textsc{paral·lel}, \textit{PARAL·LEL, paral·lel, \textsc{paral·lel}}\\{\scriptsize PARAL·LEL, paral·lel, \textit{paral·lel}, \textsc{paral·lel}}}
\paragraph*{Serbian/Macedonian} replaces б by \begin{serbian}б\end{serbian} and italic \textit{б, д, г, п, т, ш, ѓ} by \begin{serbian}\itshape б, д, г, п, т, ш, ѓ\end{serbian}

\subsection{Character Composition and Decomposition}
\paragraph*{ccmp*} contains rules for composing characters with combining diacritics like suppressing the tittle on i and j when they take an above diacritic: i + \textasciicircum → i\char"302, or to use flatter version when stacking diacritics: a\char"302\char"300, e\char"304\char"301, u\char"308\char"301\ and flatter diacritics on glyphs with ascenders: l\char"301.
\paragraph*{ss20} decomposes precomposed characters into combinations of base characters and combining diacritics (or special variants thereof) conforming to the Unicode Normalization Charts. This is useful in cases when the desired precomposed forms don’t exist in the font but their components do (see smallcaps). This feature can be used to exchange a diacritic with an alternate form (see Greek variants).

\subsection{Figures}
This font is primarily intended for texts. Thus default figures are oldstyle proportional so they fit best in a text setting. For many purposes though, tabular (all figures have the same width) and/or lining figures (they are about caps height) are preferred. They can be accessed using the features \texttt{tnum} and \texttt{lnum}.
The combinations are:
\begin{tabbing}
 Oldstyle (medieval)8888 \= 88888888888888  \= 88888888888888 \kill
  \> Proportional \> Tabular \\
 Oldstyle (medieval) \> 12345678.90 \> {\addfontfeature{RawFeature=+tnum}12345678.90} \\
 Lining		\> {\addfontfeature{RawFeature=+lnum}12345678.90} \> {\addfontfeature{RawFeature={+tnum,+lnum}}12345678.90} \\
\end{tabbing}



\subsection{Fractions}
The feature \texttt{frac} allows for arbitrary fractions: {\addfontfeature{RawFeature=+frac} 1234/567890}

\subsection{Sub- and Superscripts}
There are four positions for optically sized Sub- and Superscripts, accessed by the features \texttt{sinf}, \texttt{subs}, \texttt{ordn} and \texttt{sups}:
\paragraph*{Scientific inferiors (sinf)} l{\addfontfeature{RawFeature=+sinf}abcdefghijklmnopqrstuvwxyz0123456789}x
\paragraph*{Subscript (subs)} l{\addfontfeature{RawFeature=+subs}abcdefghijklmnopqrstuvwxyz0123456789}x
\paragraph*{Ordinals (ordn)} l{\addfontfeature{RawFeature=+ordn}abcdefghijklmnopqrstuvwxyz0123456789}x
\paragraph*{Superscript (sups)} l{\addfontfeature{RawFeature=+sups}abcdefghijklmnopqrstuvwxyz0123456789}x
\\
\setlength\subsupersep{1pt}
Combination of Sub- and Superscript: \textsubsuperscript[r]{92}{235}U
\subsection{Small Caps}
The features \texttt{smcp} and \texttt{c2sc} replace lower and uppercase glyphs by their smallcap equivalents:
\paragraph*{smcp:} \textsc{Tramway, Straßenbahn, Métro}
\paragraph*{c2sc:} {\addfontfeature{RawFeature=+c2sc}Tramway, Straßenbahn, Métro}
\paragraph*{c2sc+scmp:} \textsc{\addfontfeature{RawFeature=+c2sc}Tramway, Straßenbahn, Métro}

\subsection{Ligatures}
\paragraph*{Standard Ligatures*} (feature \texttt{liga}): f and ſ + letter with ascender and f/ſ+i get replaced by a ligature: fluffy, ſhy, official;\\
Additionally in italic: Q and g + descending character: \textit{egg, nagy, gjuha, Qyteti} 
\paragraph*{Historical Ligatures} (feature \texttt{hlig}): historical ligatures for s+t and c+t and for s+p and s+k (the latter two italic only): {\addfontfeature{RawFeature=+hlig}standard, acta, \textit{speak, skies, standard, select}}
\paragraph*{Contextual Ligatures*} (feature \texttt{clig}, italic only): ligatures for final s after a, e, i and u: \textit{%\addfontfeature{RawFeature=+clig}
finalis, ligatures, humus, rosas}
\paragraph*{Discretionary Ligatures} (feature \texttt{dlig}): T+h ligature: \addfontfeature{RawFeature=+dlig} Théatre, \textit{Theorie}

\subsection{Contextual Alternates*}
The feature \texttt{calt} controls the behaviour of contextually conditioned lettershapes. This allows to have a long-bowed \enquote{f} and \enquote{ſ} like in the original font wherever the bow doesn’t collide with the next letter or cover an empty space: gefährlich fahren ſüße Weſen;\\
The long tailed Q is managed by this feature too: Qui, Quis, Quo, \textsc{Shqipëria}, IRAQ, 

\subsection{Swash}
The italic font will have a full set of swash capitals, accessible via the \texttt{swsh}: \\\textit{\addfontfeature{RawFeature=+swsh}A B C D E F G H I J K L M N O P Q Qui R S T U V W X Y Z\\
\scriptsize A B C D E F G H I J K L M N O P Q Qu R S T U V W X Y Z}
\subsection{Character Variants and Stylistic Sets}
There are alternate representations for some glyphs, accessible via Character Variants (\texttt{cv01–cv99}) and as Stylistic Sets (\texttt{ss01–ss20}):

\subsubsection{Historical Variants}

At Claude Garamont’s times some conventions were still closer to latin orthography than to those of modern times. The following lookups imitate that usage:

\paragraph{ss02:} v and u are visual representations of the same letter dependent on initial or non initial placement: {\addfontfeature{RawFeature=+ss02}l’universe, une œuvre}
\paragraph*{cv01:} {\addfontfeature{RawFeature=+cv01}substitute non-final s with ſ (cave: this feature is \textsc{not} intelligent enough for german blackletter usage and probably other post-renaissance orthographies; it might be preferable to directly input ſ instead of using this feature!)}
\paragraph*{cv03:} j as a separate letter didn’t have a fixed place  in renaissance orthographies, so one will find i where one expects a j. This feature replaces j with i: {\addfontfeature{RawFeature=+cv03}le jour}
\paragraph*{ss05:} The Bible de Genève shows an apostrophe that sits upon the following letter if that has no ascender: {\addfontfeature{RawFeature=+ss05}l’estendue}

\subsubsection{General Variants}

\paragraph*{cv04:} Alternate ampersand (only italic): \textit{\&} → \textit{\addfontfeature{RawFeature=+cv04}\&}

\paragraph*{cv05:} Alternate forms of v (only italic): 
%% due to a bug in fontspec this doesn’t work as expected:
%\textit{v → {\addfontfeature{CharacterVariant=5:0}\addfontfeature{CharacterVariant=5:1}v}}, \textit{\addfontfeature{CharacterVariant=5:2}v}, \textit{\addfontfeature{CharacterVariant=5:3}v}
{\fontspec{EB Garamond 12 Italic} v →
\addfontfeature{CharacterVariant=5:0} v,
\addfontfeature{CharacterVariant=5:1} v,
\addfontfeature{CharacterVariant=5:2} v,
\addfontfeature{CharacterVariant=5:3} v}

\paragraph*{cv06:} Alternate forms of guillemets (narrower): {\addfontfeature{RawFeature=-cv06}« ‹ › »} → {\addfontfeature{RawFeature=+cv06}« ‹ › »}

\paragraph*{cv11:} Variants for the numeral one: some publishers prefer a numeral one that’s clearly different from small cap i: \textsc{i123} vs. {\addfontfeature{RawFeature={+cv11,+smcp}}i123}

\paragraph*{cv21} Variants for a: a vs. {\addfontfeature{RawFeature=+cv21}a}

\paragraph*{cv27} Variants for g: g/\textit{g} vs. {\addfontfeature{RawFeature=+cv27}g/\textit{g}}

\paragraph*{cv47} Variants for ß: first Variant sets all ß to the ß-shape, second to ss: {weiß {\fontspec[CharacterVariant=21:0]{EBGaramond12-Regular}weiß} {\fontspec[CharacterVariant=21:1]{EBGaramond12-Regular}weiß}}

\paragraph*{cv82} Alternate comma-shaped caron after d, l and t (for czech and slovak): ď, ľ, ť → {\addfontfeature{CharacterVariant=82}ď, ľ, ť}

\subsubsection{Symbol Variants}
\paragraph*{cv90:} floral hearts variants: \char"2619{}  and \char"2767{}  → {\addfontfeature{RawFeature=+cv90}\char"2619{} and \char"2767}

\paragraph*{cv91:} Alternate Variant for the Pilcrow in form of the old capitulum symbol: {\addfontfeature{RawFeature=+cv91}¶}

\subsubsection{Cyrillic Variants}
\paragraph*{ss01:} triangular variants of Д, Л, Љ, \textit{Д, Л, Љ}, д, л and љ (in italic uppercase only): {\addfontfeature{RawFeature=+ss01} Д, Л, Љ, \textit{Д, Л, Љ}, д, л, љ 
}
{
\subsubsection{Greek Variants}
\paragraph*{cv80:} \addfontfeature{RawFeature=+ss20}alternate circumflex: ῆ ἧ → {\addfontfeature{RawFeature=+cv80}ῆ ἧ} (\textsc{cave!} this enables complete decomposition like with ss20!)
\paragraph*{cv81:} prosgegrammeni instead of ipogegrammeni: ᾞ → {\addfontfeature{StylisticSet=20,CharacterVariant=81:1} ᾞ} → {\addfontfeature{StylisticSet=20,CharacterVariant=81:2} ᾞ} (\textsc{cave!} this enables complete decomposition like with ss20!)\\
Planned: alternate β, ε, θ, κ, π, ρ, σ, φ;\\
}
\subsubsection{IPA Variants}
\paragraph*{ss06:} variants of greek letters that fit better with the other (latin) letters used in the IPA: β, θ, λ, χ → {\addfontfeature{RawFeature=+ss06}β, θ, λ, χ}

\clearpage

%% Show the glyphs
\section{Glyphs}
\fontsize{16}{16}
\begin{center}
{\addfontfeature{LetterSpace=10.0}
 ¿ABCDEFGHIJKLMNOPQQRSẞTUVWXYZÞÐ?\\
 ¡abcdefghijklmnopqrſsſßtuvwxyzþðđf!\\
 \$0123456789€ \XeTeXglyph 99 %
 {\addfontfeature{Numbers=Lining}%
  9876543210}%
 ¢\\
 »«;:,.-–—›‹„“”‚‘’[…]\{\}()·×÷\\
 "'\#\%\& *+/<=>@\textbackslash \textasciicircum \textasciitilde §©ªº°µΩ\\
 \textsc{abcdefghijklmnopqrstuvwxyzß}\\
 {\addfontfeature{LetterSpace=0}
 fffiffiflfbfhfjfkft%\\
 ffbffhffjffkfflfft\\
 ſſſiſſiſlſbſhſjſkſt%\\
 ſſbſſhſſjſſkſſlſſt
 }\\
 \char"261E %
 {\addfontfeature{RawFeature=+hlig}ÆæstctœŒ}%
 \char"261C\\
 {\addfontfeature{LetterSpace=0}
 Text{\addfontfeature{RawFeature=+sups}abcdefghijklmnopqrstuvwxyz0123456789}x %\\
 Text{\addfontfeature{RawFeature=+ordn}abcdefghijklmnopqrstuvwxyz0123456789}x\\
 Text{\addfontfeature{RawFeature=+subs}abcdefghijklmnopqrstuvwxyz0123456789}x %\\
 Text{\addfontfeature{RawFeature=+sinf}abcdefghijklmnopqrstuvwxyz0123456789}x}\\
ÁÀÃĄÄ{\addfontfeature{Language=German,LetterSpace=10.0}Ä}%Ǎ
ÂĂÅÉÈẼ%
ĘËĚÊĔÍÌĨĮÏǏÎĬİ\\
ÓÒÕǪÖ%
{\addfontfeature{Language=German,LetterSpace=10.0}Ö}%
ǑÔŎŐØÚÙŨŲÜ%
{\addfontfeature{Language=German,LetterSpace=10.0}Ü}ǓÛŬŮŰ\\
áàãąäǎâăåéèẽęëěêĕíìĩįïǐîĭıóòõǫöǒôǒőøúùũųüǔûŭůű\\
ĆćÇçĎďĢģĜĝǦǧĞğĤĥḨḩĴĵĸŁłĽľÑñŇňŃńŊŋ\\
ŘřŔশŠšŞşŤťŢţŴŵÝýŶŷŸÿŹźŻżŽž\\
{\addfontfeature{RawFeature={+smcp,+c2sc}}%ÁÀÃĄÄÂĂÅÉÈĘËĚÊĔÍÌĨĮÏÎĬİ%\\
%ÓÒÕÖÔŎŐØÚÙŨŲÜÛŬŮŰ\\
áàãąäâăåéèęëěêĕíìĩįïîĭıóòõ%
{\addfontfeature{StylisticSet=20}ǫ}öô%
{\addfontfeature{StylisticSet=20}ǒ}őøúùũųü%
{\addfontfeature{StylisticSet=20}ǔ}ûŭůű\\
ĆćÇçĎďĢģĜĝĞğĤĥĴĵĸŁłĽľÑñŇňŃńŊŋ\\
ŘřŔশŠšŞşŤťŢţŴŵÝýŶŷŸÿŹźŻżŽž}\\
{\Huge \char"E001 \char"B6 \char"E002}\\
АБВГДЕЖЗИЙКЛМНОПРСТУФХЦЧШЩЪЫЬЭЮЯ\\
\begin{serbian}	{\addfontfeature{Language=Serbian,LetterSpace=10.0,RawFeature=+ss01}%
	АБВГДЂЕЖЗИЈКЛЉМНЊОПРСТЋУФХЦЧЏШ}
\end{serbian}\\
абвгдежзийклмнопрстуфхцчшщъыьэюя\\
\begin{serbian}
	{\addfontfeature{Language=Serbian,LetterSpace=10.0,RawFeature=+ss01}%
	абвгдђежзијклљмнњопрстћуфхцчџш}
\end{serbian}\\
\textsc{абвгд%
	{\addfontfeature{RawFeature=+ss01}д}%
	ђежзийјкл%
	{\addfontfeature{RawFeature=+ss01}л}%
	љ%
	{\addfontfeature{RawFeature=+ss01}љ}%
	мнњопрстћуфхцчџшщъыьэюя}\\
{\Huge \char"2619 {\addfontfeature{RawFeature=+cv91}¶}\char"2767}\\
ΑΒΓΔΕΖΗΘΙΚΛΜΝΞΟΠΡΣΤΥΦΧΨΩ\\
αβγδεζηθικλμνξοπρσςτυφχψω\\
ϐϝϑϰϖϱϲϡϕϻ\\
άέήίόύώϊϋΐΰ\\
\textsc{αβγδεζηθικλμνξοπρσςτυφχψω}\\
{\Huge \char"2766}\\
}
\end{center}
\clearpage
\fontsize{16}{16}\itshape
\begin{center}{\addfontfeature{LetterSpace=10.0}
¿ABCDEFGHIJKLMNOPQQRSẞTUVWXYZÞÐ?\\
{\addfontfeature{RawFeature=+swsh}ABCDEFGHIJKLMNOPQQRSTUVWXYZ}
¡abcdefghijklmnopqrsſßtuvw{\addfontfeature{RawFeature=+cv05:1}vw}{\addfontfeature{RawFeature={+ss03}}vw}xyzþðđf!\\
\$0123456789€ \XeTeXglyph 99 {\addfontfeature{Numbers=Lining}9876543210}¢\\
»«;:,.-–—›‹„“”‚‘’[…]\{\}()·×÷\\
"'\#\%\&{\addfontfeature{RawFeature=+cv04}\&}*+/<=>@\textbackslash \textasciicircum \textasciitilde §©ªº°µΩ\\
\textsc{abcdefghijklmnopqrstuvwxyzß}\\ 
fffiffiflfbfhfjfkft%\\
ffbffhffjffkfflfft\\
ſſſiſſiſlſbſhſjſkſt%\\
ſſbſſhſſjſſkſſlſſt\\
\char"261E \addfontfeature{RawFeature=+dlig}ÆæstctœŒ\char"261C\\
{\addfontfeature{LetterSpace=0}
Text{\addfontfeature{RawFeature=+sups}abcdefghijklmnopqrstuvwxyz0123456789}x\\
Text{\addfontfeature{RawFeature=+ordn}abcdefghijklmnopqrstuvwxyz0123456789}x\\
Text{\addfontfeature{RawFeature=+subs}abcdefghijklmnopqrstuvwxyz0123456789}x\\
Text{\addfontfeature{RawFeature=+sinf}abcdefghijklmnopqrstuvwxyz0123456789}x}\\
ÁÀÃĄÄ{\addfontfeature{Language=German}Ä}%Ǎ
ÂĂÅÉÈ%Ẽ
ĘËĚÊĔÍÌĨĮÏ%Ǐ
ÎĬİ\\
ÓÒÕǪÖ{\addfontfeature{%
Language=German}Ö}ǑÔŎŐØÚÙŨŲÜ{\addfontfeature{Language=German}Ü}ǓÛŬŮŰ\\
áàãąä%ǎ
âăåéè%ẽ
ęëěêĕíìĩįï%ǐ
îĭıóòõǫöǒôǒőøúùũųü%ǔ
ûŭůű\\
ĆćÇçĎďĢģĜĝǦǧĞğĤĥḨḩĴĵĸŁłĽľÑñŇňŃńŊŋ\\
ŘřŔশŠšŞşŤťŢţŴŵÝýŶŷŸÿŹźŻżŽž\\
{\addfontfeature{RawFeature={+smcp,+c2sc}}ÁÀÃĄÄÂĂÅÉÈĘËĚÊĔÍÌĨĮÏÎĬİ\\
ÓÒÕÖÔŎŐØÚÙŨŲÜÛŬŮŰ\\
áàãąäâăåéèęëěêĕíìĩįïîĭıóòõ{\addfontfeature{StylisticSet=20}ǫ}ö{\addfontfeature{StylisticSet=20}ǒ}ôǒőøúùũųü{\addfontfeature{StylisticSet=20}ǔ}ûŭůű\\
ĆćÇçĎďĢģĜĝĞğĤĥĴĵĸŁłĽľÑñŇňŃńŊŋ\\
ŘřŔশŠšŞşŤťŢţŴŵÝýŶŷŸÿŹźŻżŽž}\\
\Huge \char"E001 \char"B6 \char"E002}\\
АБВГДЕЖЗИЙКЛМНОПРСТУФХЦЧШЩЪЫЬЭЮЯ\\
абвгдежзийклмнопрстуфхцчшщъыьэюя\\
\begin{serbian}{\addfontfeature{Language=Serbian,LetterSpace=10.0,RawFeature=+ss01}
абвгдђежзијклљмнњопрстћуфхцчџш}\end{serbian}\\
ΑΒΓΔΕΖΗΘΙΚΛΜΝΞΟΠΡΣΤΥΦΧΨΩ\\
αβγδεζηθικλμνξοπρστυφχψω
\vfill
\fontsize{12}{12}\scriptsize\textit{{\addfontfeature{RawFeature=+swsh}Georg Duffner} | www.georgduffner.at/ebgaramond | g.duffner@gmail.com}
\end{center}
\end{document}